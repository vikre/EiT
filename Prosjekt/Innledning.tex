\section{Innledning}
\subsection{Introduksjon}

``Eksperter i team" er et obligatorisk fag for alle 4. klassinger ved NTNU. Formålet med faget er først og fremst å forbedre studentenes evne til samarbeid på tvers av fagområder. Alle studentene velger en landsby med et spesifikt tema. Teamene innad i en landsby settes sammen av landsbyleder med et mål om størst mulig tverrfaglighet. Vårt team består av studenter fra Datateknikk, Informatikk, Kommunikasjonsteknologi og Industriell Økonomi og Teknologiledelse og tilhører landsbyen: ``IT for en bedre verden".\\

Resultatet av prosjektet er både en prosjekt- og prosessrapport. Prosessrapporten omhandler teamets samarbeid mens prosjektrapporten presenterer våre ideer i forbindelse med vår spesifikke problemstilling. Vi har valgt å se på potensialet i forbindelse med en koblingsagent som skaper koblinger mellom to parter, først og fremst med utgangspunkt i ``Gi bort dagen". Vår problemstilling lyder som følger:
Hvilket potensiale ligger i en koblingsagent i forhold til ``Gi bort dagen" og ``Good Stuff"?

\begin{itemize}
  \item Hva vil en koblingsagent ha å si for Good Stuffs bærekraftighet?
  \item Kan en slik koblingsagent demonstreres ved hjelp av en prototype basert på enkel koblingslogikk?
  \item Hvilke fremtidige bruksmuligheter ligger i en koblingsagent på generelt basis?
\end{itemize}

Denne problemstillingen utgjør hovedfokus for innholdet i denne prosjektrapporten, og mye av prosjektarbeidet har blitt benyttet til å utvikle en prototype på en automatisk koblingsagent. Dette har resultert i at mye tid utover rapportskriving har gått med på koding til prototypen, samt å sette seg inn i ny teknologi og nye verktøy.

\subsection{Bakgrunn for oppgaven}

Landsbytemaet: ``IT for en bedre verden" er et åpent tema som gir mye rom for individuelle tolkninger. Som inspirasjon i forbindelse med utarbeidelsen av en problemstilling holdt Marianne Danielsen, daglig leder for Engasjert Byrå, en presentasjon om ``Gi bort dagen". Konseptet ``Gi bort dagen" skal motivere ressurssterke bedrifter og organisasjoner til å ``gi bort en dag" ved å knytte relasjoner med mindre ressurssterke organisasjoner som trenger hjelp av ulikt slag.

Konseptet har vært vellykket og antall relasjoner mellom mottakere og givere har økt jevnt siden oppstarten i 2012. Da antall relasjoner øker årlig kreves det stadig mer ressurser i forbindelse med relasjonsbyggingen siden dette foregår manuelt i dag. Marianne har planer om å kunne videreutvikle ``Gi bort dagen" ved å grunnlegge det ideelle selskapet ``Good Stuff" i nær fremtid. ``Good Stuff" vil være prosjekteier for ``Gi bort dagen". ``Good Stuff" skal være bærekraftig ved blant annet å tilby etisk/moralsk rådgivning og konferanser. For å oppnå dette vil det være svært gunstig med teknologi som forenkler prosessen med å koble en giver og en mottaker. Dagens opplegg med manuelle koblinger mellom mottakere og givere vil være alt for tid-og ressurskrevende dersom ideen skal utvikles videre. I den forbindelse har vi valgt å se på mulighetene en automatisk koblingsagent vil tilføre ``Gi bort dagen" i fremtiden. I hvilken grad kan en automatisk koblingsagent bidra til at ``Gi bort dagen"-prosjektet blir mer skalerbart, og samtidig bærerkraftig? I tillegg ønsker vi å se på hvilke andre bruksområder innen frivillighet som en automatisk koblingsagent kan benyttes til.

\subsection{Motivasjon}

Gruppas medlemmer hadde tilsynelatende ganske like forventninger til prosjektet. Vi forventet alle å utvikle oss selv, og da spesielt med tanke på samarbeidsevner. Et felles ønske var at vi ved prosjektslutt skulle ha bedre innsikt i hvordan man bør forholde seg til en gruppe for å oppnå en optimal prosess med et best mulig resultat. I tillegg hadde vi et ønske om å tilegne oss ny kunnskap ved utvikling av, og arbeid med koblingsagenten. Et generelt mål var å kunne utnytte hverandres tverrfaglighet til å få innsyn i fagområder vi i utgangspunktet hadde mindre innsikt i.

Landsbytemaet ``IT for en bedre verden" er implisitt en motivasjon. Det å kunne være med å utvikle noe som resulterer i en bedre verden, er noe som både inspirerte og motiverte oss. Å bidra til at vanskeligstilte og mindre ressurssterke parter får hjelp, tilfører prosjektet en ekstra dimensjon som gir deg følelsen av å være med på noe som er større enn deg selv.

\subsection{Begrensninger}

Den største utfordringen og begrensningen i prosjektet var mest sannsynlig tidsaspektet. Til sammen er det satt opp 15 landsbydager med arbeidstid på åtte timer per dag. Deler av disse dagene vil dog bli benyttet til andre øvelser, noe som fører til mindre avsatt arbeidstid til selve prosjektet. Dette er en av begrunnelsene for å utvikle en prototype istedenfor et fullstendig system. Samtidig skal arbeidet reflektere over muligheter og resultater et fullstendig system kan føre med seg ved et videre arbeid.

Videre kan kompleksitet bli et problem i prosjektet. I sammenheng med tidsaspektet diskutert over må begrensninger foretas for å kunne overkomme begge utfordringene. Noe som vil gjenspeile seg i prototypen.

Som det fremgår i introduksjonen består gruppen av personer med relativt like bakgrunner. I et prosjekt kan det ofte være fordelaktig med tverrfaglige medlemmer for å forsikre et høyt kunnskaps- og erfaringsnivå. Det er viktig at gruppa er observant på dette faktum, og baserer prosjektet på de kunnskapene og erfaringene vi besitter, men samtidig utfordrer oss selv for å unngå fenomenet gruppetenkning. Gruppetenkning er ofte et resultat av gruppemedlemmer med like interesser, egenskaper og kunnskaper, og kan føre til at første og antatt beste løsning blir valgt (Wheelan, 2009).

\subsection{Disposisjon}

I denne rapporten vil vi først presentere forarbeidet som er gjort i forbindelse med prosjektet i kapittel 2. Dette arbeidet besto blant annet av brainstorming, presentasjon og møte med Marianne Danielsen og observasjon av data, og bidro til at vi fikk et klarere bilde av krav og spesifikasjoner. Videre inneholder del 3 teori om agenter for å etablere en felles forståelse av begrepet. Delen inneholder mer spesifikke definisjoner, kategorisering og typiske egenskaper ved agenter. Deretter beskriver vi prototypen til koblingsagenten i del 4, hvordan den overordnede koblingsprosessen foregår fra brukerens perspektiv, men også hvordan den underliggende koblingslogikken fungerer. Kapittel 5 er diskusjon rundt koblingsagentens påvirkning på Good Stuff, i tillegg til videre bruksmuligheter for en generell koblingsagent. Rapporten avsluttes med resultater av en koblingagent (\ref{chapter:forslag}), konklusjon (\ref{chapter:konklusjon}) og videre arbeid (\ref{chapter:viderearbeid}).
