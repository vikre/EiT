Grunnlaget for dette prosjektet er kobling mellom giver og mottaker i forbindelse med ``Gi bort dagen" som er et initiativ startet av Marianne Danielsen ved Engasjert Byr�. Dette samsvarer godt med landsbytemaet ``IT for en bedre verden" ved at en automatisk koblingsagent vil bidra til at ``Gi bort dagen" vokser og dermed har mulighet til � hjelpe flere mennesker. V�r problemstilling dreier seg om hvilket potensiale som ligger i en koblingsagent i forhold til ``Gi bort dagen" og ``Good Stuff". Mer spesifikt hva en koblingsagent har � si for Good Stuffs b�rekraftighet, hvordan en slik koblingsagent kan representeres ved en prototype, og hvilke bruksmuligheter en koblingsagent gir p� generelt basis. 

Bakgrunnen for v�r utvikling av koblingsagentprototypen baserer seg p� presentasjon og m�te med Marianne Danielsen, en brainstormingsprosess innad i gruppa og gjennomgang av input i forbindelse med tidligere manuelle koblinger. Det er ogs� inkludert teori om agenter p� generelt grunnlag. V�r prototype presenteres b�de fra brukerens perspektiv ved � guide gjennom de fem ulike stegene koblingsprosessen best�r av, men ogs� ved � g� dypere inn i selve logikken bak koblingsprosessen. Det er identifisert flere muligheter en koblingsagent kan gi ``Good Stuff". Blant annet reduserte l�nnskostander, �kt potensiale for �konomiske bidragsytere, �kt mulighet for annonseringsinntekter, mulighet for r�dgivning gjennom identifisering av mangel p� verdier og mulighet for registreringsavgift. Disse mulighetene resulterer i at ``Good Stuffs" sannsynlighet for � v�re b�rkraftig �ker drastisk. 

I tillegg er det identifisert tre andre prosjekter som koblingsagenten kan benyttes i forbindelse med: ``Strekk ut h�nden", ``iFriend" og kobling mellom barn og aktiviteter. Dette er alle prosjekter som kan tilby en fremtidig forretnings-/utviklingsmuligheter for ``Good Stuff". 

Da ressursene tilgjengelige i prosjektarbeidet har v�rt begrensede, spesielt tid og arbeidskraft, har det v�rt n�dvendig � presentere videre arbeid i forbindelse med prototypen. Videre arbeid som er kritisk for realisering av koblingsagenten i praksis er henholdsvis etablering av en fast bruker, � inkludere delvis matching og mulighet for � kunne opprette flere ``oppdrag" for samme bruker. 
