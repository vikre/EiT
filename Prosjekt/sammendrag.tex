\section{Sammendrag}
Grunnlaget for dette prosjektet er kobling mellom giver og mottaker i forbindelse med ``Gi bort dagen" som er et initiativ startet av Marianne Danielsen ved Engasjert Byrå. Dette samsvarer godt med landsbytemaet ``IT for en bedre verden" ved at en automatisk koblingsagent vil bidra til at ``Gi bort dagen" vokser og dermed har mulighet til å hjelpe flere mennesker. Vår problemstilling dreier seg om hvilket potensiale som ligger i en koblingsagent i forhold til ``Gi bort dagen" og ``Good Stuff". Mer spesifikt hva en koblingsagent har å si for Good Stuffs bærekraftighet, hvordan en slik koblingsagent kan representeres ved en prototype, og hvilke bruksmuligheter en koblingsagent gir på generelt basis. 

Bakgrunnen for vår utvikling av koblingsagentprototypen baserer seg på presentasjon og møte med Marianne Danielsen, en brainstormingsprosess innad i gruppa og gjennomgang av input i forbindelse med tidligere manuelle koblinger. Det er også inkludert teori om agenter på generelt grunnlag. Vår prototype presenteres både fra brukerens perspektiv ved å guide gjennom de fem ulike stegene koblingsprosessen består av, men også ved å gå dypere inn i selve logikken bak koblingsprosessen. Det er identifisert flere muligheter en koblingsagent kan gi ``Good Stuff". Blant annet reduserte lønnskostander, økt potensiale for økonomiske bidragsytere, økt mulighet for annonseringsinntekter, mulighet for rådgivning gjennom identifisering av mangel på verdier og mulighet for registreringsavgift. Disse mulighetene resulterer i at ``Good Stuffs" sannsynlighet for å være bærkraftig øker drastisk. 

I tillegg er det identifisert tre andre prosjekter som koblingsagenten kan benyttes i forbindelse med: ``Strekk ut hånden", ``iFriend" og kobling mellom barn og aktiviteter. Dette er alle prosjekter som kan tilby en fremtidig forretnings-/utviklingsmuligheter for ``Good Stuff". 

Da ressursene tilgjengelige i prosjektarbeidet har vært begrensede, spesielt tid og arbeidskraft, har det vært nødvendig å presentere videre arbeid i forbindelse med prototypen. Videre arbeid som er kritisk for realisering av koblingsagenten i praksis er henholdsvis etablering av en fast bruker, å inkludere delvis matching og mulighet for å kunne opprette flere ``oppdrag" for samme bruker. 
