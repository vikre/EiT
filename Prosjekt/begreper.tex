\section*{Forklaring av begreper}

Under forklarer vi en del termer vi bruker senere i prosjektrapporten.

{\bf Boolsk verdi:} Er en måte å representere et matematisk uttrykk for om noe er sant eller usant. I programmeringsverdenen er det representert ved ``true'' eller ``false'' og 0 eller 1.\citep{website:wiki_boolean}\\

{\bf Bærekraftighet:} Å være i stand til å opprettholde et bestemt nivå.\citep{website:sustainable}\\

{\bf eVoting:} Elektronisk Stemmegivning. \citep{website:wiki_evoting}\\

{\bf PHP:} Et dynamisk skriptspråk med en god støtte opp mot databaser.\citep{website:wiki_PHP}

{\bf MySql database:} Er en relasjonsdatabase hvor man kan lagre, endre og slette data. \citep{website:glossary}\\

{\bf SMART house:} Er et hjem som er utstyrt med lys, varme og elektroniske enheter som kan bli kontrollert fra f.eks en smarttelefon eller en PC. \citep{website:smarthome}\\

{\bf Koblingsalgoritme:}  I denne teksten vil dette referere til en algoritme som vil opprette en relasjon mellom to parter ved å filtrere ut fra visse parametere.\\

{\bf Sesjon:} En sesjonsvariabel brukes til å lagre informasjon om, eller bytte instillinger for en bruker. En sesjonsvariabel holder informasjon om en enkelt bruker og er tilgjengelige på alle sider i en applikasjon.\citep{website:sessionvariable}
