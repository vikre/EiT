\section{Brainstorming}
Temaet ``IT for en bedre verden'' ga stort spillerom med tanke på problemstilling. Dette resulterte i en kreativ brainstormingprosess med svært mange ideer, alt fra avanserte tekniske ideer som SMART house til mer politisk rettede ideer som eVoting. For å begrense antall ideer reduserte vi lista fra rundt 50 til 10 ved å kun beholde de ideene gruppa som helhet var mest interessert i. Deretter ble de gjenstående forslagene grundig gjennomgått og vurdert med utgangspunkt i gruppemedlemmenes interesser og kompetanse, i tillegg til faktorer som rådighet av tid, kompleksitet, relevanse til landsbytemaet og informasjonstilgjengelighet. Etter omstendig arbeid og diskusjon ble endelig problemstilling rundt en koblingsagent valgt ved konsensus.\\

Arbeid med en koblingsagent mellom mottakere og givere i forbindelse med ``Gi bort dagen'' ga alle medlemmene mulighet til å bidra med sin kompetanse og kunnskap. Alle har erfaring med systemutvikling, og noe erfaring med databaser. Linn, Per og Mats får bidratt med sine programmeringskunnskaper, mens Espen og Silje kan bidra med sin økonomiske innsikt og kunnskap om grupper og arbeid i team. Dessuten var denne problemstillingen svært relevant i forhold til landsbytemaet og hele gruppa syns det var en spennende oppgave.\\

\section{Presentasjon av ``Gi bort dagen'' ved Marianne Danielsen}
Dag tre holdt Marianne Danielsen fra Engasjert Byrå en presentasjon om ``Gi bort dagen''. Hun presenterte bakgrunnen for ideen, og ga oss mer informasjon om konseptet og fremtidige planer. Hun satte lys på problemene med bærekraftighet ved ``Gi bort dagen'' da stadig mer arbeidstid må benyttes da konseptet blir større og flere koblinger må gjøres manuelt. Marianne presenterte forslaget om en automatisk koblingsagent som kunne spare Engasjert Byrå for mange ressurser samtidig som konseptet blir mer skalerbart. Ved å implementere en automatisk koblingsagent kan mottakere og givere selv legge inn informasjon og kontakte hverandre ettersom koblinger blir foreslått mellom egnede parter. I tillegg til å være ressursbesparende vil dette også kunne medføre tettere kontakt mellom mottaker og giver som igjen kan bidra til mer langvarige relasjoner utover ``Gi bort dagen''. Dette arbeidet er svært egnet for ``eksperter i team'' da ``Gi bort dagen'' er basert på frivillig arbeid og få IT-selskaper har tid eller mulighet til å gjøre dette arbeidet gratis.\\

Presentasjonen endte i en brainstorming der alle i landsbygruppa var deltakende. Dette ga både inspirasjon og nye ideer som vi kunne benytte videre i prosessen. Vi var blant annet innom temaer som sponsorer, donasjoner, forretningsmodell og tekniske løsninger for koblingen.

\section{Møte med Marianne Danielsen}
Dag fem hadde gruppa møte med Marianne Danielsen for å få mer input i forbindelse med brukerkrav, men også for idemyldring med tanke på alternative løsninger for det endelige resultatet. Marianne fortalte oss mer inngående om prosessen rundt den eksisterende, manuelle koblingen mellom mottakere og givere. I tillegg kom hun med forslag og ønsker til selve implementeringen. Det viste seg at Marianne ikke hadde mange tekniske krav til koblingsagenten, annet enn at den skulle være automatisk slik at hennes ansatte ikke trengte å bruke tid på manuelle koblinger. Det var også viktig for henne at ordlyden brukt i systemet var positiv og engasjerende slik at folk ble motivert og oppmuntret til å bidra. Ved å unngå formelle og kjedelige meldinger vil man kunne skape et litt annet miljø som bidrar til velvillighet og godt humør blant bidragsyterne. Marianne kom også med et forslag om å fokusere på verdiene til giverne. På denne måten kan det fremtidige ideelle selskapet “Good Stuff”, prosjekteier for ``Gi bort dagen'', identifisere mulige kunder for rådgivining innen organisasjonsverdier. På denne måten kan ``Gi bort dagen'' være bidragsytende for at ``Good Stuff'' blir et bærekraftig firma gjennom å skape flere forretningsmuligheter.

\section{Gjennomgang av tidligere input fra mottakere og givere}
I etterkant av møtet med Marianne Danielsen sendte hun oss en oversikt over informasjon innsendt av tidligere mottakere og givere til ``Gi bort dagen''. Ved å studere dataen kunne vi lettere identifisere problemområder ved de eksisterende innfyllingsskjemaene, og komme med forslag til forbedring. Dessuten var oversikten nyttig i arbeidet med å kategorisere bidragene givere ønsket å komme med, og ønskene til mottakere. Ved å foreta en slik gjennomgang og kategorisering ble det mye lettere å bestemme logikken for vår automatiske koblingsagent. Det ble blant annet bestemt at vi til størst mulig grad unngår tekstbokser, og heller benytter avkrysningsbokser slik at den tekniske koblingen mellom mottaker og giver blir mindre avansert. Dessuten bidrar dette til en enklere innfyllingsprosess for brukeren og det blir lettere å unngå misforståelser.

\section{Oppsummering}
Vi kan konkludere med at omfattende undersøkelser rundt koblingsagenter var unødvendig i dette tilfellet. Det er ingen tekniske krav til løsningen slik at vi står veldig fritt når det gjelder implementeringen av prototypen. Fokuset står på brukervennlighet og enkelhet fremfor komplekse koblingsalgoritmer.

Likevel var det viktig å få god innsikt i ``Gi bort dagen'' og ideene Marianne Danielsen hadde i forbindelse med prosjektet for å produsere et best mulig resultat som imøtekommer ønsker og forenkler brukerens opplevelse. De viktigste funnene gjort i forarbeidet  er som følger:

\begin{itemize}
    \item Det er ønsket en automatisk koblingsagent mellom mottakere og givere slik at ressursbruk reduserers og konseptet ``Gi bort dagen'' blir mer skalerbart, men samtidig bærerkraftig.
    \item En positiv ordlyd er ønsket på hjemmesiden og ved bruk av koblingsagenten slik at det skapes en god og positiv stemning.
    \item Det foreslås fokus på givende bedrifters verdier for å muliggjøre andre forretningmuligheter, som rådgvining, for ``Good Stuff''.
    \item Innfyllingsskjemaene kan simplifiseres ved å erstatte tekstbokser med kategorisering og avkrysningsbokser. Dette forenkler innfyllingsprosessen for brukere, men vil også forenkle det tekniske aspektet bak den automatiske koblingen.
\end{itemize}
