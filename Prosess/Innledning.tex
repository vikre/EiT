Gruppearbeid utgjør en stadig større del av menneskers hverdag. I løpet av en dag utgjør vi en del av flere ulike grupper i forbindelse med blant annet jobb, skole, fritidsaktiviteter, familie og venner \citep{gruppeteori}. Denne utviklingen medfører at gruppesamarbeid bør settes mer i fokus slik at man får utnyttet den mulige positive synergien ved gruppearbeid, og oppnår et optimalt resultat \citep{gruppeteori}.\\

Det er denne observasjonen som utgjør intensjonen bak faget “Eksperter i team” ved NTNU. Alle master- og profesjonsstudenter ved NTNU er pålagt å ta dette faget for å “lære samarbeid gjennom å anvende sin fagkunnskap i et tverrfaglig prosjektarbeid” \citep{website:emnebeskrivelse}. Faget setter fokus på hvordan individer med tverrfaglig bakgrunn bør oppføre seg i gruppesammenheng med tanke på å oppnå en best mulig gruppedynamikk. Ettersom gruppearbeid utgjør en stadig voksende arbeidsform, vil dette faget være en fordel for både den enkelte student, men også for næringslivet.\\

Denne prosessrapporten utgjør 50 prosent av vurderingen i faget “Eksperter i team” og hovedfokus er samarbeidet og selve prosessen rundt gruppas prosjekt. Prosessrapporten fungerer altså som et verktøy for å bevisstgjøre gruppa på ulike aspekter ved samarbeidet, både det som har fungert bra og mindre bra. En slik bevisstgjøring vil forhåpentligvis medføre at hvert enkelt medlem blir klar over hvordan han/hun bør oppføre seg i en tverrfaglig gruppe, og gjennom spesifikke situasjoner og refleksjoner klargjøre hvordan man oppnår de beste resultatene i fremtidige gruppearbeid.
