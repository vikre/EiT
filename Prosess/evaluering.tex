
Som en innledning til faget ble våre individuelle forventninger skrevet ned og presentert. Disse forventingene var mål og ambisjoner vi håpet på å oppfylle i løpet av prosjektfasen, og som alle medlemmene har etterstrebet. På slutten av prosjektet valgte gruppen å analysere utbytte av faget, og om de ulike målene og ambisjonene var oppnådd. I kapittel 5 er dette presentert grundig der alle medlemmene satt igjen med et godt inntrykk av opplevelsen EiT hadde brakt med seg. Mer spesifikt så man klare likhetstrekk mellom de individuelle analysene der alle gruppemedlemmene virket å ha blitt mer bevisst på ulike aspekter ved gruppesamarbeid.\\

Prosjektet førte også med seg ulike situasjoner i løpet av arbeidsfasen. Disse ble hovedsakelig oppdaget ved hjelp av personlige refleksjoner og grupperefleksjoner. De mest markante situasjonene var enighet, involvering og beslutningstaking, og er henholdsvis diskutert i avsnitt \ref{sec:enighet},  \ref{sec:involvering} og  \ref{sec:beslutningstaking}. Alle situasjonene har vært tilstedeværende i løpet av hele prosjektets gang, og har vært områder gruppemedlemmene og gruppa som helhet har satt mye fokus på. De ulike situasjonene har vært i konstant utvikling, der ulike aksjoner utført av gruppa har drevet situasjonene i en positiv retning, og har styrket både samhold og kvalitet på det endelige arbeidet. De ulike situasjonene har også økt bevisstheten innenfor disse spesifikke områdene til de forskjellige gruppemedlemmene.\\

Mye av prosessrapporten har fokusert på gruppas utvikling, både som helhet og på det individuelle planet. Gruppemedlemmene har gjennomført både personlige refleksjoner og grupperefleksjoner som har blitt benyttet til å analysere spesifikke hendelser, og har vært grunnlaget bak ulike tiltak. Mange av disse tiltakene har resultert i forbedringer av gruppas arbeidsmetoder, og har bidratt til økt selvinnsikt. Emnebeskrivelsen og emnets læringsmål nevner at faget skal bidra økt innsikt og utvikling av gruppemedlemmenes samarbeidsevner. Dette er noe alle medlemmene føler faget har ført med seg, og som framtidige arbeidstakere føler vi oss mer utrustet i forhold til senere prosjektarbeid i det tverrfaglige yrkeslivet.
