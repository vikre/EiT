

Denne prosessrapporten er utarbeidet som en del av faget ``Eksperter i team - IT for en bedre verden" (TDT4850) ved Norges teknisk-naturvitenskapelige universitet (NTNU). Prosessrapporten skal ta for seg ulike aspekter ved samarbeid i gruppa der mye fokus ligger på å identifisere spesifikke situasjoner som har oppstått i løpet av prosjektfasen, og trekke disse opp mot relevant teori. Det hele starter med en presentasjon av gruppesammensetningen der ulikheter og likheter blir trukket fram. Samarbeidsavtalen og ulike samarbeidsverktøy blir også presentert.\\

Videre kommer hovedkapittelet som omhandler gruppedynamikk. Her blir tre fremtredende aspekter ved gruppearbeid diskutert, henholdsvis enighet, involvering og beslutningstaking. Disse er basert på spesifikke hendelser i løpet av prosjektets gang. Kapittelet tar også for seg to spørreundersøkelser som ble utført og besvart av gruppemedlemmene under prosjektet, og vedlagt er disse undersøkelsene oppsummert i to samarbeidsindikatorer. Etter dette blir gruppetyper diskutert hvor vår gruppe blir kategorisert som en “effektiv gruppe”. Kapittelet avrundes ved å se på gruppas utvikling i løpet av prosjektet ved hjelp av Tuckmans gruppeutviklingsmodell.\\

I kapittel fire blir den personlige utviklingen til de ulike medlemmene tatt for seg i et læringsperspektiv. Her kommer de ulike medlemmene med sine tanker, og det blir konkludert med at medlemmene har hatt en positiv opplevelse av EiT og at de har blitt mer bevisste på aspekter ved gruppearbeid. Rapporten avrundes så med et evaluerings- og oppsummeringskapittel.\\

Vi vil gjerne benytte anledningen til å takke både Asbjørn Thomassen (landsbyleder) og Marianne Danielsen (daglig leder i Engasjert Byrå), samt landsbyfasilitatorene for hjelp i løpet av hele prosjektfasen.
