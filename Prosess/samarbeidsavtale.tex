\section{Samarbeidsavtale}
\label{appendix:samarbeidsavtale}
\bf{Gruppe 2 - ''IT for en bedre verden''}\\

\bf{\textit{Leveranse}}
\begin{enumerate}
\item Gruppa er enige om at alle skal bidra tilnærmet likt i arbeidet med både prosess- og prosjektrapport. Vi vil ha en overordnet ansvarlig for henholdsvis prosess- og prosjektrapporten, men alle på gruppa skal etterstrebe at rapportene er av en slik kvalitet at de tilfredsstiller kravene til en toppkarakter (A/B).
\item Alle møter til avtalt tid. Skjer det noe spesielt så skal det gis beskjed snarest mulig til hele gruppa på telefon eller Facebook om forsinkelser eller fravær. Ved mangel av beskjed blir dette tatt opp ved neste møte.
\item Etter innsjekk hver landsbydag har gruppa en “daily standup” for å kartlegge hva som er blitt gjort siden sist og sette opp mål for dagen.
\item Det forventes at alle gjør arbeidet som de har blitt tildelt til gitte tidsfrister.\\


\bf{\textit{Trivsel}}

\item Gruppa ønsker å skape en plattform for åpenhet og inkludering hvor alle har en lik mulighet til å komme med innspill.
\item For å sikre en effektiv arbeidsdag hvor motivasjonen holdes oppe så ønsker gruppa å ha effektive arbeidsøkter på ca 1 time avbrutt av en ca 5 minutters pause. I disse arbeidsperiodene skal mobiler legges vekk slik at fokuset blir opprettholdt.
\item Vi setter opp milepæler underveis i fremdriftsplanen hvor alle har ansvar for å utføre sin del av arbeidet til gitt milepæl. Gjennomført milepæl symboliseres.
\item Ved oppdagelse av avvik eller uenigheter skal dette tas opp på et tidligst mulig tidspunkt for å unngå at problemet eskalerer.
\item Dersom en konflikt går ut av kontroll skal fasilitator kontaktes for rådgivning.
Ved beslutninger skal kompromiss etterstrebes. Der dette ikke lar seg gjøres benyttes avstemning.\\

\bf{\textit{Læring}}

\item Under møter og diskusjoner så rulleres det på sekretærrollen slik at alle bidrar likt jevnt over alle møter og diskusjoner.

\item Det er viktig at man stiller (kritiske) spørsmål selv om en eller flere personer viser høy kompetanse innenfor emnet. Dette for å øke bevisstheten og kunnskapen til alle i gruppa.

\item Ved tilførsel av ny kunnskap skal man forsikre seg om at alle gruppemedlemmene er underforstått med innholdet, at det er entydig.
\end{enumerate}
